\documentclass[preprint,12pt,3p]{elsarticle}
\usepackage{amssymb}

\usepackage[utf8]{inputenc}
\journal{Journal of Applied Research and Technology}

\begin{document}
\begin{frontmatter}
	
\title{Comparison of the control algorithms associated with\\the Force Augmenting Devices}
%\tnotetext[label0]{This is only an example}
\author[FIME]{S. K. Gadi\corref{cor1}}
\address[FIME]{Facultad de Ingeniería Mecánica y Eléctrica, Universidad Autónoma de Coahuila, Torreón, Mexico.}
\cortext[cor1]{Author to whom all correspondence should be addressed.}
\ead{Research@SKGadi.com}
\ead[url]{SKGadi.com}
\author[ITA]{E. Ramírez-Velazco}
%\author[ITA]{Efrain Ramírez-Velazco}
\address[ITA]{Departamento de Eléctrica-Electrónica, Instituto Tecnológico de Aguascalientes, Aguascalientes, Mexico.}
\author[FIME]{F. E. González-Sánchez}
\author[FIME]{J. M. Sánchez-Hernández}
%\author[FIME]{Francisco Emmanuel González-Sánchez}
%\author[FIME]{José Martín Sánchez-Hernández}

\begin{abstract}
Human-robot interaction (HRI) combines the positive trades of humans and machines. A control algorithm makes sure that the machine follows the human instructions. Also, it should ensure a stable HRI. This article presents a comparison of four  HRI control algorithms. It identifies the hardware and software requirements for the algorithm's implementation. Also, it presents the pros and cons of these algorithms.
\end{abstract}

\begin{keyword}
Force Augmenting Device \sep Exoskeletons \sep Human-Robot Interaction
\end{keyword}
	
\end{frontmatter}
\section{Introduction}
\label{Introduction}
There is a growing interest in the area of human-robot interaction. These interactions are of two types: 1) The mechanical forces are not exchanged between the human and the robot arms. Example: Teleoperation 2) The robot arm and the human arm produces reaction forces on each other. This article focuses on the interaction where human and robot are in contact all the times. The human-exoskeleton interaction constitutes these interactions, where anthropomorphic robot arm moves along with the human arm. These force augmenting devices can be used in various applications ranging from active prosthetics, material handling, military, space research, etc.
\par
Since these force augmenting devices (FAD) are always in contact with the human, stability is of extreme importance. This article presents four control schemes whose stability is rigorously studied. In this document, the words exoskeletons and robot replaces the word force augmenting device.
\par
The next section presents a generalized human-FAD interaction model. Four control schemes proposed in [6]–[9] are applied to this interaction model, and their differences are studied in Section \ref{CCS}. 
\section{A general representation of a Human-Robot interaction}
\label{GHRI}
\section{Comparison of control schemes}
\label{CCS}
\section{Conclusions}
\label{Concl}

\bibliographystyle{elsarticle-num}
\bibliography{refs}
\end{document}
